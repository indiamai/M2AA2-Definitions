\documentclass{article}
\usepackage[utf8]{inputenc}
\usepackage{amsmath}

\title{M2AA2 Definitions}
\author{indiamarsden }
\date{March 2019}
\newcommand{\f}{\mathcal{F}}

\begin{document}

\maketitle
\section*{List of definitions}
\begin{enumerate}
    \item Kronecker delta
    \item Permutation symbol
    \item Directional Derivative
    \item Div, Curl
    \item Tangent plane
    \item Irrotational
    \item Solenoidal
    \item Path integral
    \item Projection Theorem
    \item Volume integral
    \item Green's Theorem
    \item Stoke's Theorem
    \item Divergence Theorem
    \item Flux
    \item Green's First Identity
    \item Green's Second Identity
    \item Flux theorem
    \item Gradient in coordinate systems
    \item Divergence in coordinate systems
    \item Cartesian Length scales
    \item Cylindrical Length scales
    \item Spherical length scales
    \item Fourier series
    \item Parseval's Theorem
    \item Half range fourier cosine series
    \item Half range fourier sine series
    \item Parsevals for half range
    \item Exponential form of fourier series
    \item Fourier transform
    \item Inverse fourier transform
    \item Fourier cosine transform
    \item Inverse Fourier cosine transform
    \item Fourier sine transform
    \item Inverse Fourier sine transform
    \item Properties of fourier transforms
    \item Properties of half range fourier transforms
    \item Convolution theorem
    \item Energy theorem
    \item MVT for integrals
    \item Dirac delta
    \item Sifting property of dirac delta
    \item Wave Equation
    \item Heat Equation
    \item Laplace's Equation
    \item Green's function in 3D
    \item Green's function in 2D
\end{enumerate}
\section{Vector Calculus}
\textbf{Kronecker delta}
$  \delta_{xy} = \begin{cases} 1 & x = y \\ 0 & x \ne y \end{cases}$
\\
\\
\textbf{Permutation symbol} $  \epsilon_{ijk} = \begin{cases} 0 & \mbox{if any two of } i, j, k \mbox{ are the same} \\ 1 & \mbox{if } i, j, j \mbox{ is a cyclic permutation of 1,2,3}  \\ -1 & \mbox{if } i, j, j \mbox{ is an acyclic permutation of 1,2,3} \end{cases}$
\\
\\
\textbf{Directional derivative}
$$ \frac{\partial \phi}{\partial s} = \frac{\partial \phi}{\partial n} \cos \theta = \frac{\partial \phi}{\partial n} (\Hat{n} \cdot \Hat{s}) = \Hat{s} \cdot \nabla \phi$$
\textbf{Divergence} $ \mbox{div } A = \nabla \cdot A$
\\
\\
\textbf{Curl} $ \mbox{curl } A = \nabla \times A$
\\
\\
\textbf{Tangent plane }$(\boldsymbol{r} - \boldsymbol{r}_P)\cdot(\nabla \phi)_P = 0$
\\
\\
\textbf{Sum and product formulae}
\begin{align*}
    \nabla (\phi_1 + \phi_2) &= \nabla \phi_1 + \nabla \phi_2 \\
    \mbox{div} (A + B) &= \mbox{div} A + \mbox{div} B
    \\
    \mbox{curl} (A+B) &= \mbox{curl} A + \mbox{curl} B
    \\
    \nabla (\phi \psi) &= \phi \nabla \psi + \psi \nabla \phi
    \\
    \mbox{div} \phi A &= \phi \mbox{div} A + \nabla \phi \cdot A
\end{align*}
\textbf{Laplacian}
$$\nabla^2 A = \frac{\partial^2 \boldsymbol{A}}{\partial x^2} + \frac{\partial^2 \boldsymbol{A}}{\partial y^2} + \frac{\partial^2 \boldsymbol{A}}{\partial z^2}$$
\textbf{Curl of a gradient } $\mbox{curl } \nabla \phi = 0$
\\
\\
\textbf{Divergence of a curl} $ \mbox{div (curl} A) = 0$
\\
\\
\textbf{Curl of a curl} $\mbox{curl(curl(} A) = \nabla (\mbox{div }A) - \nabla^2 A$
\\
\\
\textbf{Irrotational Vector field} Curl A = 0
\\
\\
\textbf{Solenoidal Vector field} Div A = 0
\\
\\
\textbf{Path integral } $$\int_\gamma A \cdot \textit{d}r = \int_\gamma (A \cdot \frac{\textit{d}r}{\textit{d}s}) \textit{d}s$$
\\
\\
\textbf{Conservative forces} Where $F = \nabla \phi$, $\int_\gamma A \cdot \textit{d}r  = \phi(B) - \phi(A)$, which is independent of the path joining A and B.
\\
\\
\textbf{Potential} In the case of conservative field F = $\nabla \phi$, the differentiable, scalar, function $\phi$ is called the potential
\\
\\
\textbf{Closed surface} this divides three-dimensional space into two non-connected regions - an interior region and an exterior region
\\
\\
\textbf{Simple closed surface} this is a closed surface which does not intersect itself
\\
\\
\textbf{Convex surface} this is a surface which is crossed by a straight line at most twice
\\
\\
\textbf{Open surface} this does not divide space into two non-connected regions - it has a boundary which can
be represented by a closed curve. (A closed surface can be thought of as the sum of two open surfaces).
\\
\\
\textbf{Projection Theorem}
$$
\int_S \textit{f}(P) \textit{d}S = \int_\Sigma \textit{f}(P) \frac{\textit{d}x\textit{d}y}{| n \cdot \Hat{k} |} 
$$
\\
\\
\textbf{Volume Integral}
$$
\int_\tau f d \tau = \int_\tau f dxdydz
$$
in Cartesian coordinates.
\\
\\
\textbf{Green's Theorum}
$$
\oint_C (L  dx + M dy) = \int_R (\frac{\partial M}{\partial x} - \frac{\partial L}{\partial y}) dx dy
$$ where R is a closed plane region bounded by a simple convex anticlockwise curve C in the x-y plane.
Also true for non convex curves by splitting region into convex sections and summing. Additionally, it can be extended to multiply connected regions by interpreting C to be the entire, outer and inner boundary, with the region always to the left.
\\
\\
\textbf{Stokes Theorem}
$$
\oint_C \mathbf{F} \cdot d \mathbf{r} = \int_R \text{curl }  \mathbf{F} \cdot \hspace{2pt} d \mathbf{S}
$$
Where the direction of the curve C and the normal to S are related by a right hand rule, and S is an \textbf{open} surface bounded by C.
\\
\\
\textbf{Divergence theorem}
$$ 
\int_S \mathbf{A} \cdot \hspace{2pt} d\mathbf{S} = \int_\tau \text{div} \hspace{2pt} \mathbf{A} \hspace{2pt} d\tau
$$
if $\tau$ is a volume enclosed by a closed surface S. This can be extended to simply and multiply connected regions similarly to Green's Theorem
\\
\\
\textbf{Flux}
The flux of a surface A is $$ \int_S \mathbf{A} \cdot d \mathbf{S}$$
\\
\\
\textbf{Green's First Identity}
$$
\int_S \bigg \{ \phi \frac{\partial \psi}{\partial n}\bigg \} dS = \int_\tau \big \{ \phi \nabla^2 \psi + (\nabla \phi) \cdot (\nabla \psi) \big \} d\tau
$$ This is obtained from considering $A = \phi \nabla \psi $ and the divergence theorem.
\\
\\
\textbf{Green's Second Identity}
$$
\int_S \bigg \{ \phi \frac{\partial \psi}{\partial n} - \psi \frac{\partial \phi}{\partial n}\bigg \} dS = \int_\tau \big \{ \phi \nabla^2 \psi - \psi \nabla^2 \phi \big \} d\tau
$$ which can be derived from Green's first identity, interchanging the symbols and subtracting one from the other. 
\\
\\
\textbf{Gauss's flux theorem}
$$
\int_S \frac{\mathbf{\hat{n} \cdot r}}{r^3} dS = \begin{cases} 0, & \text{if the origin is exterior to S} \\ 4\pi, & \text{if the origin is interior to S}\end{cases}
$$
\textbf{Inverse function theorem}
Around any point where $det(J(x_u))$ is non zero, there exists a neighbourhood in which the $u_i$ can be expressed as single valued function of the $x_i$.
\\
\\
\textbf{Curvilinear coordinate system} Define $u_i = u_i(x_1, x_2, x_3)$ as a single valued function. $(u_1, u_2, u_3$ is said to be a curvilinear coordinate system. 
\\
\\
\textbf{Orthogonal curvilinear coordinate system}
Define $a_i = \frac{\nabla u_i}{| \nabla u_i |}$. If $(a_1, a_2, a_3)$ are mutually orthogonal (ie $a_i \cdot a_j = 0 \forall i \ne j$) then it is an orthogonal curvilinear coordinate system.
\\
\\
\textbf{Length scales}
\\
$$
h_i = \bigg | \frac{\partial \mathbf{r}}{\partial u_i} \bigg |
$$
\\
\\
\textbf{Path element} 
$$
d\mathbf{r} = h_1du_1\hat{\mathbf{e}}_1 + h_2du_2\hat{\mathbf{e}}_2 + 
h_3du_3\hat{\mathbf{e}}_3
$$
\\
\textbf{Volume element}
$$
d\tau = h_1h_2h_3du_1du_2du_3
$$
\\
\textbf{Surface element}
\\
With $u_1$ constant:
$$
dS = h_2h_3du_2du_3
$$
\textbf{Gradient in coordinate systems}
$$
\nabla \Phi = \frac{\hat{\mathbf{e}}_1}{h_1} \frac{\partial \Phi}{\partial u_1} +  \frac{\hat{\mathbf{e}}_2}{h_2} \frac{\partial \Phi}{\partial u_2} +  \frac{\hat{\mathbf{e}}_3}{h_3} \frac{\partial \Phi}{\partial u_3}
$$
\\
\textbf{Divergence in orthogonal coordinate systems}
$$
\nabla \cdot A = \frac{1}{h_1h_2h_3} \bigg \{ \frac{\partial}{\partial u_1}(h_2h_3A_1) +\frac{\partial}{\partial u_2}(h_1h_3A_2) + \frac{\partial}{\partial u_3}(h_1h_2A_3) \bigg \}
$$
\textbf{Curl in orthogonal coordinate systems}
$$
curl A = \frac{1}{h_1h_2h_3} \begin{vmatrix} h_1\hat{\mathbf{e}}_1 & h_2\hat{\mathbf{e}}_2 & h_3\hat{\mathbf{e}}_3 \\
\frac{\partial}{\partial u_1} & \frac{\partial}{\partial u_2} & \frac{\partial}{\partial u_3}  \\ h_1A_1 & h_2A_2 & h_3A_3\end{vmatrix}
$$
\\
\textbf{Laplacian in orthogonal coordinate systems}
\begin{align*}
\nabla^2 A &= \nabla \cdot \nabla A \\
&= \frac{1}{h_1h_2h_3} \bigg \{ \frac{\partial}{\partial u_1}(h_2h_3\frac{1}{h_1}\frac{\partial A}{\partial u_1}) +\frac{\partial}{\partial u_2}(h_1h_3\frac{1}{h_2}\frac{\partial A}{\partial u_2}) + \frac{\partial}{\partial u_3}(h_1h_2\frac{1}{h_3}\frac{\partial A}{\partial u_3}) \bigg \}
\end{align*}
\textbf{Cartesian Coordinates} $(x,y,z)$ 
\\
\begin{align*}
    &d\tau = dxdydz \\
    &d\mathbf{r} = dx\mathbf{i}  + dy\mathbf{j} + dz\mathbf{k}\\
    &h_1 = h_2 = h_3 = 1 \\
    &\nabla \Phi = \mathbf{i} \frac{\partial \Phi}{\partial x} +  \mathbf{j} \frac{\partial \Phi}{\partial y} +  \mathbf{k} \frac{\partial \Phi}{\partial z}
\end{align*}
\\
\textbf{Cylindrical Polar coordinates} $(r, \theta, z)$
\\
\begin{align*}
    &d\tau = rdrd\theta dz \\
    &d\mathbf{r} = dx\mathbf{i}  + rd\theta \mathbf{j} + dz\mathbf{k}\\
    &h_1 = h_3 = 1 \\
    & h_2 = r \\
    & \nabla \Phi = \hat{\mathbf{r}} \frac{\partial \Phi}{\partial r} +  \frac{\hat{\theta}}{r} \frac{\partial \Phi}{\partial \theta} +  \hat{\mathbf{z}} \frac{\partial \Phi}{\partial z}
\end{align*}
\\
\textbf{Spherical Polar coordinates} $(r, \theta, \phi)$
\\
\begin{align*}
    &d\tau = r^2sin(\theta)drd\theta d\phi \\
    &d\mathbf{r} = dx\mathbf{i}  + rd\theta \mathbf{j} + rsin(\theta)d\phi \mathbf{k}\\
    &h_1  = 1 \\
    & h_2 = r \\
    & h_3 = rsin(\theta) \\
    & \nabla \Phi = \hat{\mathbf{r}} \frac{\partial \Phi}{\partial r} +  \frac{\hat{\theta}}{r} \frac{\partial \Phi}{\partial \theta} +  \frac{\hat{\phi}}{rsin(\theta)} \frac{\partial \Phi}{\partial \phi}
\end{align*}
\textbf{Changes of Variable in surface integration} \\
Say we have a surface parameterized by two variables ie
$$
x = m(u,v), y = g(u,v), z= h(u,v)
$$
Then we can write the surface element as:
$$
dS = \bigg | \frac{\partial \mathbf{r}}{\partial u} \times \frac{\partial \mathbf{r}}{\partial v} \bigg | dudv
$$
and then:
$$
\int_S f(x,y,z) dS = \int_S F(u,v) \bigg | \frac{\partial \mathbf{r}}{\partial u} \times \frac{\partial \mathbf{r}}{\partial v} \bigg | dudv
$$
where $F(u,v) = f(m(u,v),g(u,v),h(u,v))$

\section{Fourier Series}

\textbf{Orthonormal systems}\\
A sequence of integrable functions $\{ \phi_i \}_{i=1}^\infty$ on an interval [a,b] is called orthonormal if:
$$
\int_a^b \phi_i\phi_j dx = \delta_{ij}
$$
\\
\textbf{Periodic functions} \\
A function F is periodic with period a if:
$$
F(x) = F(x+a) \text{for all x}
$$
\textbf{Even and Odd functions}\\
A function is \textbf{even} about x=a iff $f(x+a) = f(x-a)$ for all x \\
A function is \textbf{odd} about x=a iff $f(x+a) = -f(x-a)$ for all x
\\
\\
\textbf{Fourier Series} \\
Take $f(x)$ a periodic function with period 2L. We define the Fourier series over $[-L,L]$.
$$
f(x) = \frac{1}{2}a_0 + \sum_{n=1}^\infty a_n \text{cos}(\frac{m\pi x}{L}) + b_n \text{sin}(\frac{m\pi x}{L})
$$
with coefficients defined as:
\begin{align*}
    a_n &= \frac{1}{L} \int_{-L}^L f(x)\text{cos}(\frac{n\pi x}{L}) dx 
    & x = 0 \to \infty \\
    b_n &= \frac{1}{L} \int_{-L}^L f(x)\text{sin}(\frac{n\pi x}{L}) dx 
    & x = 1 \to \infty \\
\end{align*}
\\
\textbf{Parseval's Theorem} \\
If we define f(x) as a Fourier series as above then:
$$
\frac{1}{L}\int_{-L}^L (f(x))^2 dx = \frac{1}{2}a_0^2 + \sum_{n=1}^\infty (a_n^2 + b_n^2)
$$
\\
\textbf{Half-range Fourier cosine series}
\\
\begin{align*}
    f(x) &= \frac{1}{2}a_0 + \sum_{n=1}^\infty a_n\text{cos}(\frac{n\pi x}{L}) & (0 \leq x \leq L) \\
    a_n &= \frac{2}{L} \int_0^L f(x)\text{cos}(\frac{n\pi x}{L}) dx & (n = 0 \to \infty)
\end{align*}
\textbf{Half-range Fourier Sine series}
\\
\begin{align*}
    f(x) &= \sum_{n=1}^\infty a_n\text{sin}(\frac{n\pi x}{L}) & (0 \leq x \leq L) \\
    a_n &= \frac{2}{L} \int_0^L f(x)\text{sin}(\frac{n\pi x}{L}) dx & (n = 1 \to \infty)
\end{align*}
\\
\textbf{Parseval's for half range series}
\begin{align*}
    \frac{2}{L}\int_{0}^L (f(x))^2 dx \hspace{5pt} &= \frac{1}{2}a_0^2 + \sum_{n=1}^\infty a_n^2 & \text{(Cosine series)}\\
    &=  \sum_{n=1}^\infty  b_n^2 & \text{(Sine series)}
\end{align*}
\\
\textbf{Exponential form of the Fourier Series}
\\
\begin{align*}
    f(x) &= \sum_{n=-\infty}^\infty c_n e^{\frac{in\pi x}{L}} & |x| < L \\
    c_n &= \frac{1}{2L}\int_{-L}^{L} f(x) e^{-\frac{in\pi x}{L}} dx, & n = -\infty \to \infty
\end{align*}
\section{Fourier transforms}
\textbf{Fourier transform}
$$
\hat{f}(\omega) = \int_{-\infty}^\infty f(x)e^{-i\omega x}dx
$$
\textbf{Inverse Fourier Transform}
$$
f(x) = \frac{1}{2\pi}\int_{-\infty}^\infty \hat{f}(\omega)e^{i\omega x}d\omega
$$
\textbf{Fourier cosine transform}
$$
\hat{f}_c(\omega) = \int_{0}^\infty f(x)\text{cos}(\omega x)dx
$$
Note for even f(x): $\hat{f}(\omega) = 2\hat{f}_c(\omega)$ \\
\textbf{Inverse Fourier cosine transform}
$$
 f(x)= \frac{2}{\pi}\int_{0}^\infty \hat{f}_c(\omega)\text{cos}(\omega x)d\omega
$$
\textbf{Fourier sine transform}
$$
\hat{f}_s(\omega) = \int_{0}^\infty f(x)\text{sin}(\omega x)dx
$$
\textbf{Inverse Fourier sine transform}
$$
\hat{f}_s(\omega) = \frac{2}{\pi} \int_{0}^\infty f(x)\text{sin}(\omega x)d\omega
$$
\subsection{Properties of Fourier Transforms}
\begin{enumerate}
    \item $\mathcal{F} \{ af(x) +bg(x) \} = a\hat{f}(\omega) + b\hat{g}(\omega) $ \hspace{6pt} (Linearity)
    \item $\f \{ f(ax) \} = \frac{1}{a} \hat{f}(\frac{\omega}{a}) $ \hspace{6pt} For a $>$ 0
    \item $\f \{ f(x-x_0) \} = e^{-i\omega x_0} \hat{f} (\omega)$
    \item $\f \{e^{i\omega_0 x}f(x) \} = \hat{f}(\omega-\omega_0) $
    \item $\f \{ \hat{f} (x) \} = 2\pi f(-\omega)$ \hspace{6pt} Symmetry Formula
    \item $\f \{ d^n f / d x^n \} = (i\omega)^n \hat{f}(\omega) $
    \item $\f \{ xf(x) \} = i \hat{f}' (\omega) $
\end{enumerate}
\subsection{Properties of Half range Fourier Transforms}
\begin{enumerate}
    \item $\mathcal{F}_c \{f'(x) \} = -f(0) + \omega \hat{f}_s(\omega)$
    \item $\mathcal{F}_s \{f'(x) \} = - \omega \hat{f}_c(\omega)$ 
    \item $\mathcal{F}_c \{f''(x) \} = -f'(0) - \omega^2 \hat{f}_c(\omega)$
    \item $\mathcal{F}_s \{f''(x) \} = \omega f(0) - \omega^2 \hat{f}_s(\omega)$
\end{enumerate}
\textbf{Convolution Theorem}
Define convolution as:
$$
f(x) * g(x) = \int_{-\infty}^{\infty} g(u)f(x-u) du
$$
Then we have the result for fourier transforms:
$$
\f \{ f * g \} = \hat{f}(\omega)\hat{g}(\omega)
$$
\textbf{Energy Theorem} \\
This is Parseval's theorem for Fourier transforms, if f(x) is  a real valued function then:
$$
\frac{1}{2\pi} \int_{-\infty}^\infty \big | \hat{f}(\omega) \big |^2 d\omega = \int_{-\infty}^\infty \big [ f(x) \big ]^2 dx
$$
\textbf{Mean value theorem for integrals} \\
For g(x) continuous on [a,b] then,
$$
\int_a^b g(x) dx = (b-a) g(\Bar{x})
$$
for some $\Bar{x}$

\textbf{Dirac Delta Function} \\
The dirac delta is defined to essentially be infinite at x = 0 and zero elsewhere. The important property is:
$$
\int_{-\infty}^{\infty} \delta(x) dx = 1
$$
\textbf{Sifting property of the Dirac Delta function}
$$
\int_{-\infty}^{\infty} g(x) \delta(x - a) dx = g(a)
$$
\section{Partial Differential equations}

\textbf{The wave equation}
$$
\frac{\partial^2 u}{\partial t^2} = c^2 \nabla^2 u
$$
\textbf{The heat equation}
$$
\frac{\partial u}{\partial t} = k^2 \nabla^2 u
$$
\textbf{Laplace's Equation}
$$
\nabla^2 \phi = f(\mathbf{r})
$$
\textbf{Green's function in 3D}
\begin{align*}
    \nabla^2 G &= \delta(\mathbf{ r}- \mathbf{r_0}) \\
    G(\mathbf{r};\mathbf{r_0}) &= - \frac{1}{4\pi |\mathbf{r} - \mathbf{r_0}|}
\end{align*}
\textbf{Green's function in 2D}
\begin{align*}
    \nabla^2 G &= \delta(\mathbf{ r}- \mathbf{r_0}) \\
    G(\mathbf{r};\mathbf{r_0}) &= \frac{1}{2\pi}\text{ln}| \mathbf{x} - \mathbf{x_0}|
\end{align*}
\end{document}
